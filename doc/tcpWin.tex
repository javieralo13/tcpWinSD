\IfFileExists{t2doc.cls}{
    \documentclass[documentation]{subfiles}
}{
    \errmessage{Error: could not find 't2doc.cls'}
}

\begin{document}

% Declare author and title here, so the main document can reuse it
\trantitle
    {tcpWin} % Plugin name
    {tcpWin} % Short description
    {Tranalyzer Development Team} % author(s)

\section{tcpWin}\label{s:tcpWin}

\subsection{Description}
The tcpWin plugin analyzes ...

\subsection{Dependencies}

%\traninput{file} % use this command to input files
%\traninclude{file} % use this command to include files

%\tranimg{image} % use this command to include an image (must be located in a subfolder ./img/)

\subsubsection{External Libraries}
This plugin depends on the {\bf XXX} library.
\paragraph{Arch:} {\tt sudo pacman -S XXX}
\paragraph{Ubuntu/Kali:} {\tt sudo apt-get install XXX}
\paragraph{OpenSUSE:} {\tt sudo zypper install XXX}
\paragraph{Red Hat/Fedora:} {\tt sudo yum install XXX}
\paragraph{MAC OSX:} {\tt brew install XXX}

\subsubsection{Other Plugins}
This plugin requires the \tranrefpl{tcpFlags} plugin.

\subsubsection{Required Files}
The file {\tt file.txt} is required.

\subsection{Configuration Flags}
The following flags can be used to control the output of the plugin:
\begin{longtable}{lcll}
    \toprule
    {\bf Name} & {\bf Default} & {\bf Description} & {\bf Flags}\\
    \midrule\endhead%
    {\tt TCPWIN\_VAR}   & 0 & Whether (1) or not (0) to output {\tt tcpWinVar}\\
    {\tt TCPWIN\_IP}    & 0 & General description of {\tt TCPWIN\_IP}\\
                        &   & \qquad 0: description of value 0\\
                        &   & \qquad 1: description of value 1\\
                        &   & \qquad 2: description of value 2\\
    {\tt TCPWIN\_VEC}   & 0 & Description of {\tt TCPWIN\_VEC} & {\tt\small TCPWIN\_IP=1}\\ % only relevant if TCPWIN_IP=1
    \bottomrule
\end{longtable}

\subsection{Flow File Output}
The tcpWin plugin outputs the following columns:
\begin{longtable}{llll}
    \toprule
    {\bf Column} & {\bf Type} & {\bf Description} & {\bf Flags}\\
    \midrule\endhead%
    {\tt \nameref{tcpWinStat}}              & H8 & Status                       & \\
    {\tt \hyperref[tcpWinStat]{tcpWinText}} & S  & describe tcpWinText (string) & \\

    {\tt tcpWinVar} & U64 & describe tcpWinVar (uint64) & {\tt\small TCPWIN\_VAR=1}\\  % only output if TCPWIN_VAR1=1
    {\tt tcpWinIP}  & IP4 & describe tcpWinIP (IPv4)    & {\tt\small TCPWIN\_IP=1} \\  % only output if TCPWIN_IP=1
    {\tt \hyperref[tcpWinVar1]{tcpWinVar1\_Var2}} & H32\_H16 & describe {\tt tcpWinVar1\_Var2} & \\
    \\
    \multicolumn{4}{l}{If {\tt TCPWIN\_VEC=1}, the following columns are displayed:}\\
    \\
    {\tt tcpWinVar3\_Var4} & R(U8\_U8) & describe {\tt tcpWinVar1\_Var2} & \\
    {\tt tcpWinVector}     & R(R(D))   & describe {\tt tcpWinVector}     & \\
    \bottomrule
\end{longtable}

\subsubsection{tcpWinStat}\label{tcpWinStat}
The {\tt tcpWinStat} column is to be interpreted as follows:
\begin{longtable}{rl}
    \toprule
    {\bf tcpWinStat} & {\bf Description}\\
    \midrule\endhead%
    {\tt 0x0\textcolor{magenta}{1}} & Flow is tcpWin\\
    {\tt 0x0\textcolor{magenta}{2}} & ---\\
    {\tt 0x0\textcolor{magenta}{4}} & ---\\
    {\tt 0x0\textcolor{magenta}{8}} & ---\\
    \bottomrule
\end{longtable}

\subsubsection{tcpWinVar1}\label{tcpWinVar1}
The {\tt tcpWinVar1} column is to be interpreted as follows:\\
\begin{minipage}{.45\textwidth}
    \begin{longtable}{rl}
        \toprule
        {\bf tcpWinVar1} & {\bf Description}\\
        \midrule\endhead%
        {\tt 0x0\textcolor{magenta}{1}} & ---\\
        {\tt 0x0\textcolor{magenta}{2}} & ---\\
        {\tt 0x0\textcolor{magenta}{4}} & ---\\
        {\tt 0x0\textcolor{magenta}{8}} & ---\\
        \bottomrule
    \end{longtable}
\end{minipage}
\hfill
\begin{minipage}{.45\textwidth}
    \begin{longtable}{rl}
        \toprule
        {\bf tcpWinVar1} & {\bf Description}\\
        \midrule\endhead%
        {\tt 0x\textcolor{magenta}{1}0} & ---\\
        {\tt 0x\textcolor{magenta}{2}0} & ---\\
        {\tt 0x\textcolor{magenta}{4}0} & ---\\
        {\tt 0x\textcolor{magenta}{8}0} & ---\\
        \bottomrule
    \end{longtable}
\end{minipage}

\subsection{Packet File Output}
In packet mode ({\tt --s} option), the tcpWin plugin outputs the following columns:
\begin{longtable}{llll}
    \toprule
    {\bf Column} & {\bf Type} & {\bf Description} & {\bf Flags}\\
    \midrule\endhead%
    {\tt tcpWinCol1} & I8 & describe col1 & \\
    \bottomrule
\end{longtable}

\subsection{Plugin Report Output}
The following information is reported:
\begin{itemize}
    \item Number of XXX packets
    \item Aggregated status flags ({\tt\nameref{tcpWinStat}})
\end{itemize}

\subsection{Additional Output}
Non-standard output:
\begin{itemize}
    \item {\tt PREFIX\_suffix.txt}: description
\end{itemize}

\subsection{Post-Processing}

\subsection{Example Output}

\subsection{Known Bugs and Limitations}

\subsection{TODO}
\begin{itemize}
    \item TODO1
    \item TODO2
\end{itemize}

\subsection{References}
\begin{itemize}
    \item \href{https://tools.ietf.org/html/rfcXXXX}{RFCXXXX}: Title
    \item \url{https://www.iana.org/assignments/}
\end{itemize}

\end{document}
